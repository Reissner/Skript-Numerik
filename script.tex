\RequirePackage[orthodox]{nag}

\documentclass[index=totoc]{scrartcl}%{article}

\usepackage{geometry}
\usepackage{microtype}
\usepackage[utf8]{inputenc}
\usepackage[T1]{fontenc}
\usepackage[ngerman]{babel}
\usepackage{fancyhdr}
\usepackage{fancyvrb}
\usepackage{caption3}
\usepackage[pdftex]{graphicx}
\usepackage{lmodern}
\usepackage{amsmath}
\usepackage{amssymb}
\usepackage{amsthm}
\usepackage{srcltx}%\usepackage{url}
\usepackage{rerunfilecheck}


\usepackage{glossaries}
\usepackage{makeidx}

% to be loaded last 
\usepackage[pdftex,bookmarks=true,bookmarksnumbered=true,hypertexnames=false,breaklinks=true,linkbordercolor={0 0 1},verbose]
{hyperref}
\hypersetup{pdfauthor={Ernst Reissner},pdftitle={Nuetzliche Mathematik},pdfsubject={LA Analysis und Numerik},bookmarksopen=true}

\pdfadjustspacing=1

\setcounter{secnumdepth}{3}
\setcounter{tocdepth}{3}
\makeindex

\newtheorem{thm}{Satz}[section]
\newtheorem{lem}[thm]{Lemma}
\newtheorem{cor}[thm]{Korollar}

\theoremstyle{definition}
\newtheorem{defi}{Definition}[section]
\newtheorem{bem}[defi]{Bemerkung}
\newtheorem{bemn}[defi]{Bemerkungen}
\newtheorem{bsp}[defi]{Beispiel}
\newtheorem{bspe}[defi]{Beispiele}
\newtheorem{afg}[defi]{Aufgabe}

\newcommand{\N}{\mathbb N}
\newcommand{\Z}{\mathbb Z}
\newcommand{\Q}{\mathbb Q}
\newcommand{\R}{\mathbb R}
\newcommand{\C}{\mathbb C}
\newcommand{\Hd}{\mathbb H}


\DeclareMathOperator{\spann}{span}

\title{Nützliche Mathematik}
\author{Ernst Reißner}

\begin{document}
%\frontmatter
\maketitle
\pdfbookmark[1]{Contents}{toc}

\tableofcontents

\section{Einführung}

Dieser Artikel ist eine Sammlung von mathematischem Hintergrund,
den wir mal für konkrete Projekte gebraucht haben,
also eigentlich aus der Numerischen Mathematik.
Erst einmal ist es eine lockere Sammlung von Sätzen.

Nach ein paar Worten zu Mengen in Abschnitt \ref{sec:setsBasics}
beginnen wir in Abschnitt \ref{sec:laBasics}
mit Grundlagen aus der Linearen Algebra,
fügen dann in Abschnitt \ref{sec:anaBasics} die Grundlagen aus der Analysis an,
immer nur das was wir wirklich brauchen.

Alle weiteren Abschnitte behandeln je ein Thema aus der Numerik.
Derzeit haben wir nur einen Abschnitt:
den über Iterationsverfahren.

\newpage
\section{Grundlagen Mengenlehre}
\label{sec:setsBasics}

Erst mal geht es da um Mengen, beginnend mit der leeren Menge. 
Wir sollten da $\N$, $\Z$ und $\Q$ einführen.
Da gehören auch Relationen dazu.
Auch Abbildungen und Familien werden gebraucht. 

\section{Grundlagen Lineare Algebra}
\label{sec:laBasics}

In der linearen Algebra geht es im Wesentlichen um Vektorräume
das sind Mengen mit deren Elementen, den Vektoren, man rechnen kann 
und um diverse (lineare) Abbildungen zwischen den Vektorräumen,
die diese Rechnungen "`erhalten"'.

In der Numerik relevante Beispiele sind der $\R^n$
sowie Funktionenräume, wie der Raum der Polynome
oder der Raum der stetigen Funktionen.
In Vektorräumen kann man rechnen, wie man es im $\R^n$ gewöhnt ist. 

Um Vektorräume verstehen zu können, müssen wir erst Gruppen behandeln, 
da Vektorräume auch Gruppen sind.
Zudem können Vektoren auch mit Skalaren multipliziert werden,
wobei diese Skalare einen Körper bilden.
Im Wesentlichen kommen nur die Körper $\R$ der reellen Zahlen
und $\C$ der komplexen Zahlen vor.
Wir betrachten auch noch die Quaternionen $\Hd$,
die einen Schiefkörper aber keinen Körper bilden.



\subsection{Gruppen, Körper und Schiefkörper}
\label{sec:grpField}

\begin{defi}\index{Verknüpfung}
  Eine {\em Verknüpfung} auf einer Menge $M$
  ist eine Abbildung  $\circ\colon M\times M\to M$,
  die man in Infix-Schreibweise $m\circ m'$ notiert,
  statt in Präfix-Schreibweise $\circ(m, m')$.

  Auch Abbildungen $\circ\colon M\times M'\to M''$
  mit verschiedenen Mengen $M$, $M'$, $M''$ werden als Verknüpfung geschrieben.

  Als Verknüpfungszeichen sind neben $\circ$,
  die Zeichen $\cdot$ und $+$ am beliebtesten. 
  Wo möglich läßt man die das Zeichen $\cdot$
  für multiplikative Verknüpfung weg,
  also schreibt man $ab$ statt $a\cdot b$.

  Die Auswertungsreihenfolge wird über Klammern beschrieben.
  Treten die Zeichen $\cdot$ und $+$ in einem Ausdruck auf,
  so kann man mit der Regel "`Punkt vor Strich"' Klammern auch weglassen.
  Bei gleichen Zeichen
  kann man die Klammern mit der Regel "`links nach Rechts"' weglassen. 
\end{defi}

\begin{bspe}
  Hier drei sehr unterschiedliche Verknüpfungen:
  %
  \begin{itemize}
  \item
  Die Addition $+$ und die Multiplikation $\cdot$ auf den reellen Zahlen $\R$
  sind Verknüpfungen.
  \item
  Ist $X$ eine Menge und $M$ die Menge der Abbildungen $X\to X$,
  dann ist die Hintereinanderausführung $\circ$ eine Verknüpfung auf $M$.
  \item
    Die Skalarmultiplikation auf $\R^n$ ist eine Verknüpfung
    %
    \begin{eqnarray*}
      \R\times\R^n&\to&\R^n \\
      \alpha\cdot
      \left(\begin{array}{c}x_1\\\vdots\\x_n\end{array}\right)
                  &=&
      \left(
      \begin{array}{c}\alpha\cdot x_1\\\vdots\\\alpha\cdot x_n\end{array}
      \right).
    \end{eqnarray*}
  \end{itemize}
\end{bspe}

\begin{defi}[Gruppe]
  \index{Gruppe}\index{Gruppe!kommutative}\index{Gruppe!Abel'sche}
  \index{Element!inverses} \index{Element!negatives}\index{Element!neutrales}
  \index{assoziativ}\index{kommutativ}
  Eine {\em Gruppe} $(G,\circ)$ ist eine Menge $G$
  versehen mit einer Verknüpfung $\circ\colon G\times G\to G$
  so dass gilt
  %
  \begin{itemize}
  \item
    Es gibt ein Element $e\in G$, so dass gilt $e\circ g=g\circ e=g$
    für alle $g\in G$.
    Dabei heißt $e$ das {\em neutrale Element} von $G$.
  \item
    Zu jedem $g\in G$ gibt es ein Element $g'\in G$
    so dass gilt $g\circ g'=g'\circ g=e$.
    Dieses Element heißt das {\em inverse Element} zu $g$. 
  \item
    Für beliebige $g,g', g''\in G$ gilt
    $(g\circ g')\circ g''=g\circ (g'\circ g'')$.
    Man sagt $\circ$ ist {\em assoziativ}. 
  \end{itemize}

  Eine Gruppe heißt {\em kommutativ} oder {\em Abel'sch},
  wenn $g\circ g'=g'\circ g$ gilt für alle $g, g'\in G$.

  Wenn die Verknüpfung $\cdot$ geschrieben wird,
  dann spricht man von einer {\em multiplikativen Gruppe}.
  Das neutrale Element heißt dann {\em Einselement} und wird als $1$ geschrieben
  und das zu $g\in G$ inverse Element wird dann als $g^{-1}$ geschrieben.

  Nur für kommutative Gruppen wird die Verknüpfung $+$ geschrieben.
  Dann spricht man von einer {\em additiven Gruppe}.
  Das neutrale Element heißt dann {\em Nullelement} und wird als $0$ geschrieben
  und das zu $g\in G$ inverse Element heißt dann {\em negatives Element}
  und wird dann als $-g$ geschrieben.
\end{defi}

\begin{bem}
  Das neutrale Element einer Gruppe $(G, \cdot)$ ist immer eindeutig,
  denn für zwei neutrale Elemente $e,e'\ G$ gilt $e=ee'=e'$.
  
  Das inverse des neutralen Elements ist das neutrale Element selbst.
  Auch das inverse Element ist eindeutig bestimmt.
  %Das linksinverse ist gleich dem rechtsinversen. 
\end{bem}

\begin{bspe}
  Folgende sind die wichtigsten Beispiele für Gruppen. 
  \begin{itemize}
  \item
    Die kleinste Gruppe ist die, die nur ein neutrales Element enthält:
    $(\{1\}, \cdot)$ bzw. $(\{0\}, +)$.
    Diese Gruppe ist natürlich kommutativ.
    Die leere Menge ist keine Gruppe. 
  \item
    Eine Gruppe  $(G, \cdot)$ mit zwei Elementen muss neben dem neutralen
    ein weiteres Element $g\in G$ enthalten, das eben nicht neutral ist.
    Das inverse zu $g$ kann nicht $1$ sein, muss also $g$ selbst sein:
    $gg=1$.
    Damit ist die Gruppe und die Verknüpfung eindeutig festgelegt
    und natürlich ist sie kommutativ.
    Diese Gruppe ist in $\R$ eingebettet als $(\{\pm1\},\cdot)$. 
  \item
    Die $(\R,+)$ ist eine (kommutative) additive Gruppe und
    $(\R\setminus\{0\},\cdot)$ ist eine multiplikative,
    ebenfalls kommmutative Gruppe.
    Das gleiche gilt noch für $\C$.
  \item
    $(\Z,+)$ ist die kleinste Gruppe, die $\N$ enthält;
    insbesondere ist $\N$ keine Gruppe. 
  \item
    $(\Q,+)$ und $(\Q\setminus\{0\}, \cdot)$ sind auch kommutative Gruppen.
  \item
    Definiert man die Addition auf $\R^n$ komponentenweise,
    dann ist auch  $(\R^n,+)$ ist eine (kommutative) additive Gruppe. 
  \item
    Ist $X$ irgendeine Menge und
    $F(X,\R^n)$ die Menge aller Funktionen $X\to\R^n$
    und $+$ die punktweise Addition, also für $f,g\in F(X,\R^n)$ sei
    %
    \begin{eqnarray*}
      (f+g)(x)&=&f(x)+g(x)
    \end{eqnarray*}
    dann ist auch $(F(X, R^n), +)$ eine Gruppe. 
  \item
    Ist $G$ die Menge der Funktionen $X\to X$,
    wobei $X$ eine beliebige Menge sei, und bezeichne $\circ$
    die Hintereinanderaussführung,
    dann bildet $(G, \circ)$ eine Gruppe,
    die im allgemeinen nicht kommutativ ist. 
    Für spezielles $X$ oder für geeignete Teilmengen $G$
    der Funktionen $X\to X$ kann die Gruppe wieder kommutativ sein. 
  \end{itemize}
\end{bspe}


\begin{bem}
  Ist $(G, \circ)$ eine Gruppe, deren Verknüpfung klar ist,
  dann sagt man schlicht, dass $G$ eine Gruppe ist.
  Das ist bei $\R^n$ klar und auch oft bei Funktionengruppen. 
\end{bem}


\begin{defi}
  \index{Körper}
  \index{distributiv}
  Ein {\em Körper} $(K, +, \cdot)$ ist eine Menge $K$,
  versehen mit zwei Verknüpfungen $+$ und $\cdot$, beide $K\times K\to K$,
  so dass
  %
  \begin{itemize}
  \item
    $(K,+)$ ist eine (kommutative) Gruppe, 
  \item
    $(K\setminus\{0\},\cdot)$ ist eine kommutative Gruppe,
  \item
    es gilt das {\em Distributivgesetz}
    %
    \begin{eqnarray*}
      a\cdot(b+c) &=& a\cdot b+a\cdot c. 
    \end{eqnarray*}
  \end{itemize}
\end{defi}

\begin{bemn}
  Man beachte, dass die multiplikative Gruppe ohne $0$ ist.
  Es gilt nämlich nach Korollar \ref{cor:factor0K} $0x=0$ für alle $x\in K$. 
  Die $0$ hat also kein multiplikatives Inverses.
  
  In der numerischen Mathematik kommen eigentlich nur zwei Körper vor:
  $\R$ und $\C$.
  Es gibt noch den Hamiltonschen Schiefkörper $\Hd$,
  aber das ist kein Körper, nur ein sogenannter Schiefkörper.
  Schiefkörper betrachten wir hier nicht.

  In allen Fällen sind die Verknüpfungen klar,
  so dass man einfach vom Körper $\R$ oder $\C$ spricht.
\end{bemn}

\begin{cor}\label{cor:factor0K}
  Ist $(K, +, \cdot)$ ein Körper, dann gilt für $a,b\in K$
  %
  \begin{eqnarray*}
    ab=0 & \iff & a=0\text{ oder }b=0. 
  \end{eqnarray*}
\end{cor}

\begin{proof}
  Es gilt
  %
  \begin{equation*}
    0b+0b=(0+0)b=0b=0+0b\text{ und damit }0b=0.
  \end{equation*}
  %
  und analog $a0=0$.
  Ist umgekehrt $ab=0$ und $b\not=0$,
  dann kann man mit dem Inversen von $b$ multiplizieren
  und erhält $a=a1=abb^{-1}=0b^{-1}=0$.
  (Analog: Ist $a\not=0$, so ist $b=0$, was aber nicht bewiesen werden muss. )
\end{proof}



\subsection{Vektorräume}
\label{subsec:vectorspace}

Die zentralen Mengen um die es in der linearen Algebra geht,
sind Vektorräume.


\begin{defi}[Vektorraum]
  \index{Vektorraum}
  \index{Vektor}
   \index{Nullvektor}
  \index{Skalar}
  \index{Vektormultiplikation}
  \index{Skalarmultiplikation}
  Sei $(V,+)$ eine (kommutative) Gruppe, $(K,\oplus,\odot)$ ein Körper
  und $\cdot\colon K\times V\to V$ eine weitere Verknüpfung so dass gilt: 
  %
  \begin{itemize}
  \item
    $\alpha\cdot(u+v)=\alpha\cdot u+\alpha\cdot v$
    für $\alpha\in K$, $u,v\in V$
  \item 
    $(\alpha\oplus \beta)\cdot v=\alpha\cdot v+\beta\cdot v$
    für $\alpha,\beta\in K$, $v\in V$
  \item
    $(\alpha\odot\beta) \cdot v = \alpha \cdot (\beta \cdot v)$
    für $\alpha,\beta\in K$, $v\in V$
  \item 
    $1 \cdot v = v$ für $1\in K$, $v\in V$ (Neutralität des Einselements)
  \end{itemize}

  Dann heißt $(V,+,\cdot)$ ein {\em $K$-Vektorraum},
  ein {\em Vektorraum über $K$}, oder kurz ein {\em Vektorraum}.

  Die Elemente von $K$ heißen {\em Skalare},
  die Elemente von $V$ heißen {\em Vektoren}.
  Vektoren werden mit lateinischen Buchstaben geschrieben,
  Skalare hingegen mit griechischen.
  Man unterscheidet zwischen dem Skalar $0\in K$
  und dem {\em Nullvektor} $0\in V$. 
  Die Addition im Vektorraum heißt {\em Vektoraddition},
  die Multiplikation von Skalar und Vektor heißt {\em Skalarmultiplikation}. 
\end{defi}

\begin{bem}
  In aller Regel schreibt man sowohl die Vektoraddition
  als auch die addition im Körper als $+$,
  und entsprechend sowohl die Skalarmultiplikation
  als auch die Multiplikation im Körper als $\cdot$.
  Wie die Bedingungen in der Definition illustrieren,
  geht die Verknüpfung aus dem Kontext hervor. 
\end{bem}

\begin{cor}\label{cor:factor0V}
  Ist $(V, +, \cdot)$ ein Vektorraum über einem Körper $K$,
  dann gilt für $\alpha\in K$ und $v\in V$
  %
  \begin{eqnarray*}
    \alpha v=0 & \iff & \alpha=0\text{ oder }v=0. 
  \end{eqnarray*}
\end{cor}

\begin{proof}
  Der Beweis ist formal wie der von Korrolar \ref{cor:factor0K},
  bis auf den letzten Satz, der aber nicht benötigt wird. 
\end{proof}

\begin{bspe}
  Sei immer $K=\R$. Der weitaus wichtigste $\R$-Vektorraum für uns ist $\R^n$,
  wobei die Vektoraddition und die Skalarmultiplikation
  komponentenweise über Adddition und Multiplikation auf $\R$ definiert sind.
  %
  \begin{eqnarray*}
    \left(\begin{array}{c}x_1    \\\vdots\\x_n    \end{array}\right)+
    \left(\begin{array}{c}    y_1\\\vdots\\    y_n\end{array}\right)=
    \left(\begin{array}{c}x_1+y_1\\\vdots\\x_n+y_n\end{array}\right)
    &\text{ und }&
    \alpha\cdot
    \left(
    \begin{array}{c}            x_1\\\vdots\\           x_n\end{array}\right)=
    \left(
    \begin{array}{c}\alpha\cdot x_1\\\vdots\\\alpha\cdot x_n\end{array}\right).
  \end{eqnarray*}

  Auch die Menge $C([0,1])$ der stetigen Funktionen $[0,1]\to\R$
  bildet einen Vektorraum,
  wenn Vektoraddition (Funktionsaddition) und Skalarmultiplikation
  punktweise durch Addition und Multiplikation auf $\R$  definiert sind:
  % 
  \begin{eqnarray*}
    (f+g)(x) = f(x)+g(x)
    &\text{ und }&
    (\alpha\cdot f)(x)=\alpha\cdot(f(x)). 
  \end{eqnarray*}

  Man kann auch $\C$ als Vektorraum über $\R$ betrachten.
  Die Vektoraddition ist die Addition in $\C$
  und die Skalarmultiplikation ist die Multiplikation
  mit einem Faktor in $\R$ und dem anderen in $\C$. 
\end{bspe}

\begin{bemn}
  Ist $(V, +, \circ)$ ein Vektorraum, dessen Verknüpfungen klar sind,
  dann sagt man schlicht, dass $V$ ein Vektorraum ist.
  Das ist bei $\R^n$ klar und auch oft bei Funktionengruppen. 
\end{bemn}

\begin{defi}
  \index{Untervektorraum}
  Sei $K$ ein Körper und $(V,+,\cdot)$ ein $K$-Vektorraum.
  Eine Teilmenge $U$ von $V$ heißt ein {\em Untervektorraum}, wenn
  %
  \begin{itemize}
  \item
    $0\in U$ für den Nullvektor $0\in V$ 
  \item
    $u+v\in U$ für $u,v\in U$ 
  \item
     $\alpha\cdot u\in U$ für $\alpha\in K$, $u\in U$ 
   \end{itemize}
   %
  d.h. man kann die Verknüpfungen auf $U$ einschränken.
  Damit ist $(U,+,\cdot)$ selbst ein $K$-vektorraum.
\end{defi}

\begin{defi}
  \index{Linearkombination}
  \index{Aufspann}
  \index{lineare Hülle}
  Sei $K$ ein Körper, $V$ ein $K$-Vektorraum und $U\subset V$ eine Teilmenge.

  Eine {\em Linearkombination} von Vektoren aus $U$ ist ein Ausdruck
  %
  \begin{equation*}
    \alpha_1\cdot v_1+\ldots+\alpha_n\cdot v_n
  \end{equation*}
  %
  mit $n\in\N_0$, $\alpha_1,\dots,\alpha_n\in K$ und $v_1,\dots,v_n\in U$.

  Der {\em Aufspann} $\spann U$ (die {\em lineare Hülle}) von $U$
    ist die Menge aller Vektoren,
  die sich als Linearkombinationen aus $U$ darstellen lassen. 

  Ist $U_I$ eine Familie von Vektoren in $V$,
  so ist der {\em Aufspann} $\spann U_I$ definiert als $\spann\{u_i|i\in I\}$.
\end{defi}

\begin{bem}
  Man beachte, dass Linearkombinationen immer endlich sind,
  auch wenn unendlich viele Vektoren zur Verfügung stehen.
  Ausserdem kann eine Linearkombination auch leer sein.
  Wegen $0=0+x$ steht die leere Linearkombination für den Nullvektor. 
\end{bem}


\begin{lem}
  Sei $K$ ein Körper, $V$ ein $K$-Vektorraum und $U\subset V$ eine Teilmenge.
  Der Aufspann $\spann U$ ist der kleinste Unterraum von $V$, der $U$ enthält. 
\end{lem}

\begin{proof}
  Erst mal ist klar, dass $U\subseteq \spann U$,
  denn alle $u\in U$ lassen sich als $1\cdot u\in\spann U$ schreiben.
  
  Dann ist auch klar, dass auch $0\in \spann U$ ist,
  sogar wenn $U$ leer ist, denn dann nimmt man die leere Linearkombination
  mit $n=0$.
  Die anderen Eigenschaften eines Unterraumes sind klar.

  Dass $\spann U$ der kleinste derartige Unterraum ist,
  geht daraus hervor, dass eine Menge die $U$ enthält
  und ein Untervektorraum ist,
  auch alle Linearkombinationen enthalten muss. 
\end{proof}

\begin{defi}
  \index{linear unabhängig}
  \index{unabhängig!linear}
  Sei $K$ ein Körper, $V$ ein $K$-Vektorraum.
  Eine Teilmenge $U\subset V$ heißt {\em linear unabhängig},
  wenn für Linearkombinationen 
  % 
  \begin{eqnarray*}
    0&=&\alpha_1\cdot v_1+\ldots+\alpha_n\cdot v_n
  \end{eqnarray*}
  %
  aus $U$ gilt $\alpha_1=\ldots=\alpha_n=0$. 
\end{defi}

\begin{cor}
  Sei $K$ ein Körper und $V$ ein $K$-Vektorraum.
  Eine Teilmenge $U\subset V$ ist linear unabhängig genau dann,
  wenn sich kein $u\in U$ als Linearkombination anderer Vektoren aus $U$
  darstellen läßt. 
\end{cor}

\begin{proof}
  Sei $U$ linear unabhängig und $u\in U$.
  Jede Linearkombination $u=\alpha_1\cdot v_1+\ldots+\alpha_n\cdot v_n$
  aus $U$ mit $v_1,\dots,v_n\not=u$ kann man umformen in 
  $0=\alpha_1\cdot v_1+\ldots+\alpha_n\cdot v_n+(-1) u$
  
  und damit $0=\alpha_1\cdot v_1+\ldots+\alpha_n\cdot v_n+(\alpha-1) u$.
  Daraus folgt $\alpha_1,\dots,\alpha_n=0$ und $\alpha=1$. 
\end{proof}


\begin{defi}
  \index{Erzeugendensystem}
  \index{Basis}
  Sei $K$ ein Körper, $V$ ein $K$-Vektorraum. 
  Eine Familie von $U_I$ in $V$ heißt ein {\em Erzeugendensystem}
  von $V$, wenn $V=\spann U_I$ ist. 
  Ein unverkürzbares Erzeugendensystem,
  also eines so dass
  %
  \begin{equation*}
      I'\subsetneq I \implies\spann U_{I'}\subsetneq\spann U_I
  \end{equation*}
  %
  gilt, heißt {\em Basis} von $V$. 
\end{defi}

\newpage

\section{Grundlagen Analysis}
\label{sec:anaBasics}

In der Analysis geht es im Wesentlichen um metrische Räume,
also um Mengen deren Elemente für die ein Abstand,
eine "`Metrik"' definiert ist,
sowie um Abbildungen, die diese Abstände "`erhalten"',
wie die stetigen Abbildungen.
Genauer definiert man über Abstände die Konvergenz von Folgen
und stetige Abbildungen sind Abbildungen, die die Konvergenz erhalten. 

Die metrischen Räume der Analysis sind oft auch Vektorräume, 
die wir in Abschnitt \ref{sec:laBasics} darstellen, 
so dass sich analytische und algebraische Vorstellungen ergänzen. 
Die auf Vektorräumen relevanten Metriken sind die Normen.





\section{Iterationsalgorithmen}\label{sec:iter}


\subsection{Der Banach'sche Fixpunktsatz }

Um nichtlineare Probleme zu lösen, gibt es fast nur Iterationsverfahren.
Die meisten davon basieren auf dem Banch'schen Fixpunktsatz.
Um ein Problem dem zugänglich zu machen,
muss man das Problem erst mal in ein Fixpunktproblem umwandeln.
Also Beispiel kann man das Nullstellenproblem $f(x)=0$
ganz naiv in ein Fixpunktproblem umwandeln durch
$F(x):=x+f(x)=x$: Nullstellen von $f$ sind gerade Fixpunkte von $F$. 

\begin{thm}[Banach'scher Fixpunktsatz]
  Sei $(X, \|.\|)$ ein vollständiger normierter Raum,
  $f\colon X\to X$ eine kontrahierende Abbildung mit Faktor $K\in[0, 1)$, d.h.
  %
  \begin{equation}\label{eq:contr}
    \|f(x)-f(x')\| \le K\,\|x-x'\| \qquad\text{für alle } x,x'\in X.     
  \end{equation}

  Dann konvergiert die Folge $(x_n)_{n\in\N}$
  definiert durch $x_{n+1}:=f(x_n)$ für $n\in\N$,
  egal welchen Startpunkt $x_0\in X$ man wählt
  gegen einen Punkt $\tilde x\in X$.

  Es gilt $f(\tilde x)=\tilde x$, d.h. $\tilde x$ ist Fixpunkt
  und  sogar der einzige Fixpunkt von $f$.

  Zudem gilt:
  %
  \begin{equation}\label{eq:BanachEstApriori}
    \|\tilde x- x_j\| \le \|x_1-x_0\|\,\frac{K^j}{1-K}
    \qquad\text{für alle } j\in\N.     
  \end{equation}
    %
  \begin{equation}\label{eq:BanachEstAposteriori}
    \|\tilde x- x_j\| \le \|x_j-x_{j-1}\|\,\frac{K}{1-K}
    \qquad\text{für alle } j\in\N.     
  \end{equation}
\end{thm}

\begin{bem}[Anwendung: Iterationsverfahren]
Der Satz klingt schon sehr abstrakt,
ist aber die Grundlage fast jeden Iterationsverfahrens,
insbesondere des Newton-Verfahrens.

Wir interpretieren den Satz jetzt in diesem Zusammenhang.
Wie oben erwähnt,
muss man das gegebene Problem in ein Fixpunktproblem umwandeln,
also die Gestalt $f(x)=x$ geben, und zwar für kontrahierendes $f$.
Die Kontraktion verringert Abstände und führt so zur Konvergenz
und es ist schon sehr intuitiv, dass der Konvergenzpunkt Fixpunkt ist.
Angenehm ist, dass der Schätzwert $x_0$
gar nicht so schlecht sein kann, dass das Verfahren nicht konvergiert. 

Spannend ist auch, dass man abschätzen kann,
wie weit man vom Fixpunkt noch entfernt ist:
Ungleichung (\ref{eq:BanachEstApriori}) zeigt, dass der erste Schritt bereits
eine Abschätzung gibt, wie groß der Fehler im $j$ten Schritt noch ist.
In aller Regel wendet man aber Gleichung (\ref{eq:BanachEstAposteriori}) an,
weil man die neuesten Erkenntnisse beim Iterieren nutzen will.

Diese Abschätzungen geben eine Abbruchbedingung für das Iterationsvefahren.
Ist man sich der Konvergenz nicht sicher,
sollte man einen Timeout einfügen. 
\end{bem}

\begin{bem}[Erläuterung]
  Kontraktion und Lipschitzstetigkeit, C1,
  Vollständigkeit, Cauchyfolge
\end{bem}

\begin{bem}[Verallgemeinerungen]
Die Iterationsfolge $(x_n)_{n\in\N}$
ist natürlich nur definiert, falls $f$ wieder nach $X$ abbildet.
Oft wendet man den Satz auch auf eine Teilmenge $X'\subseteq X$ an
mit $f(X')\subseteq X'$.
Das folgt nicht aus der Kontraktionseigenschaft,
sondern muss gesondert nachgewiesen werden.
Auch muss dann natürlich $x_0\in X'$ sein. 

Streng genommen, muss $X$ kein normierter Raum sein,
es genügt ein metrischer Raum.
Wichtig ist nur die Vollständigkeit. 
\end{bem}


\begin{proof}%[Beweis durch Spielerei]
  Wir schätzen erst mal Abstände benachbarter Folgenglieder ab:
  %
  \begin{align*}
    \|x_2-x_1\| &= \|f(x_1)-f(x_0)\|\le K\,\|x_1-x_0\| \\
    \|x_3-x_2\| &=  \|f(x_2)-f(x_1)\|\le K\,\|x_2-x_1\|\le K^2\,\|x_1-x_0\| \\
                & \dots\\
    \|x_{i+1}-x_i\| &=   K^i\,\|x_1-x_0\|\qquad\text{für alle }i\in\N_0
  \end{align*}

  Jetzt schätzen wir Abstände beliebiger Folgeglieder ab:
  Seien $i,j\in\N$ und erst mal $i>j$. Dann gilt
  %
  \begin{align*}
    \|x_i-x_j\| &= \|x_i-x_{i-1}+x_{i-1}+\ldots -x_{j+1}+x_{j+1}-x_j\| \\
                &= \|x_i-x_{i-1}\|+x_{i-1}+\ldots -x_{j+1}\|+\|x_{j+1}-x_j\| \\
                &= \sum_{k=j}^{i-1}\|x_k-x_{k-1}\| \\
                &\overset{s.o.}= \sum_{k=j}^{i-1}K^k\|x_1-x_0\| 
                = \|x_1-x_0\|K^j\sum_{k=0}^{i-j-1}K^k \\
                &\overset*= \|x_1-x_0\|K^j\frac{1-K^{i-1}}{1-K} 
                \le \|x_1-x_0\|K^j\frac1{1-K}.
  \end{align*}
  Beim Stern haben wir die Geometrische Reihe benutzt. %****
  Das gilt offensichtlich für $i=j$ und wegen $\|x_i-x_j\|=\|x_j-x_i\|$
  auch für $j>i$, also insgesamt für alle $i,j\in\N_0$.

  Diese Abschätzung zeigt, dass $\|x_i-x_j\|$ gegen null geht,
  wenn nur $i,j$ groß genug sind.
  Mit anderen Worten, es ist eine Cauchy-Folge.
  Da $X$ vollständig ist, gibt es einen Grenzwert $\tilde x\in X$
  und es gilt sogar
  %
  \begin{eqnarray*}
    \|\tilde x-x_j\|=\lim_{i\to\infty}\|x_i-x_j\|
    &\le& \|x_1-x_0\|K^j\frac1{1-K}.
  \end{eqnarray*}
  Das schätzt ab, wie gut die Konvergenz ist.

  Da $f$ Lipschitz-Stetig und damit stetig ist, % ****
  haben wir
  % 
  \begin{align*}
    f(\tilde x) &=  f(\lim_{i\to\infty}x_i)=\lim_{i\to\infty}f(x_i) \\
                &= \lim_{i\to\infty}x_{i+1}=\tilde x
  \end{align*}
  und somit ist $\tilde x$ ein Fixpunkt.

  Einen weiteren $x'\not=\tilde x$ kann es nicht geben, denn dann wäre
  %
  \begin{eqnarray*}
    \|\tilde x-x'\|=\|f(\tilde x)-f(x')\|&\le& K\|\tilde x-x'\|. 
  \end{eqnarray*}
  Aus den Eigenschaften einer Norm $\|.\|$ folgt $\|\tilde x-x'\|\not=0$
  was zu $1\le K$ führen würde, ein Widerspruch.
\end{proof}

\printindex


\end{document}
